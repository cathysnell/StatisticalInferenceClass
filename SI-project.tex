\documentclass[]{article}
\usepackage{lmodern}
\usepackage{amssymb,amsmath}
\usepackage{ifxetex,ifluatex}
\usepackage{fixltx2e} % provides \textsubscript
\ifnum 0\ifxetex 1\fi\ifluatex 1\fi=0 % if pdftex
  \usepackage[T1]{fontenc}
  \usepackage[utf8]{inputenc}
\else % if luatex or xelatex
  \ifxetex
    \usepackage{mathspec}
  \else
    \usepackage{fontspec}
  \fi
  \defaultfontfeatures{Ligatures=TeX,Scale=MatchLowercase}
\fi
% use upquote if available, for straight quotes in verbatim environments
\IfFileExists{upquote.sty}{\usepackage{upquote}}{}
% use microtype if available
\IfFileExists{microtype.sty}{%
\usepackage{microtype}
\UseMicrotypeSet[protrusion]{basicmath} % disable protrusion for tt fonts
}{}
\usepackage[margin=1in]{geometry}
\usepackage{hyperref}
\hypersetup{unicode=true,
            pdftitle={Statistical Inference Class Project},
            pdfauthor={Cathy Snell},
            pdfborder={0 0 0},
            breaklinks=true}
\urlstyle{same}  % don't use monospace font for urls
\usepackage{color}
\usepackage{fancyvrb}
\newcommand{\VerbBar}{|}
\newcommand{\VERB}{\Verb[commandchars=\\\{\}]}
\DefineVerbatimEnvironment{Highlighting}{Verbatim}{commandchars=\\\{\}}
% Add ',fontsize=\small' for more characters per line
\usepackage{framed}
\definecolor{shadecolor}{RGB}{248,248,248}
\newenvironment{Shaded}{\begin{snugshade}}{\end{snugshade}}
\newcommand{\KeywordTok}[1]{\textcolor[rgb]{0.13,0.29,0.53}{\textbf{#1}}}
\newcommand{\DataTypeTok}[1]{\textcolor[rgb]{0.13,0.29,0.53}{#1}}
\newcommand{\DecValTok}[1]{\textcolor[rgb]{0.00,0.00,0.81}{#1}}
\newcommand{\BaseNTok}[1]{\textcolor[rgb]{0.00,0.00,0.81}{#1}}
\newcommand{\FloatTok}[1]{\textcolor[rgb]{0.00,0.00,0.81}{#1}}
\newcommand{\ConstantTok}[1]{\textcolor[rgb]{0.00,0.00,0.00}{#1}}
\newcommand{\CharTok}[1]{\textcolor[rgb]{0.31,0.60,0.02}{#1}}
\newcommand{\SpecialCharTok}[1]{\textcolor[rgb]{0.00,0.00,0.00}{#1}}
\newcommand{\StringTok}[1]{\textcolor[rgb]{0.31,0.60,0.02}{#1}}
\newcommand{\VerbatimStringTok}[1]{\textcolor[rgb]{0.31,0.60,0.02}{#1}}
\newcommand{\SpecialStringTok}[1]{\textcolor[rgb]{0.31,0.60,0.02}{#1}}
\newcommand{\ImportTok}[1]{#1}
\newcommand{\CommentTok}[1]{\textcolor[rgb]{0.56,0.35,0.01}{\textit{#1}}}
\newcommand{\DocumentationTok}[1]{\textcolor[rgb]{0.56,0.35,0.01}{\textbf{\textit{#1}}}}
\newcommand{\AnnotationTok}[1]{\textcolor[rgb]{0.56,0.35,0.01}{\textbf{\textit{#1}}}}
\newcommand{\CommentVarTok}[1]{\textcolor[rgb]{0.56,0.35,0.01}{\textbf{\textit{#1}}}}
\newcommand{\OtherTok}[1]{\textcolor[rgb]{0.56,0.35,0.01}{#1}}
\newcommand{\FunctionTok}[1]{\textcolor[rgb]{0.00,0.00,0.00}{#1}}
\newcommand{\VariableTok}[1]{\textcolor[rgb]{0.00,0.00,0.00}{#1}}
\newcommand{\ControlFlowTok}[1]{\textcolor[rgb]{0.13,0.29,0.53}{\textbf{#1}}}
\newcommand{\OperatorTok}[1]{\textcolor[rgb]{0.81,0.36,0.00}{\textbf{#1}}}
\newcommand{\BuiltInTok}[1]{#1}
\newcommand{\ExtensionTok}[1]{#1}
\newcommand{\PreprocessorTok}[1]{\textcolor[rgb]{0.56,0.35,0.01}{\textit{#1}}}
\newcommand{\AttributeTok}[1]{\textcolor[rgb]{0.77,0.63,0.00}{#1}}
\newcommand{\RegionMarkerTok}[1]{#1}
\newcommand{\InformationTok}[1]{\textcolor[rgb]{0.56,0.35,0.01}{\textbf{\textit{#1}}}}
\newcommand{\WarningTok}[1]{\textcolor[rgb]{0.56,0.35,0.01}{\textbf{\textit{#1}}}}
\newcommand{\AlertTok}[1]{\textcolor[rgb]{0.94,0.16,0.16}{#1}}
\newcommand{\ErrorTok}[1]{\textcolor[rgb]{0.64,0.00,0.00}{\textbf{#1}}}
\newcommand{\NormalTok}[1]{#1}
\usepackage{graphicx,grffile}
\makeatletter
\def\maxwidth{\ifdim\Gin@nat@width>\linewidth\linewidth\else\Gin@nat@width\fi}
\def\maxheight{\ifdim\Gin@nat@height>\textheight\textheight\else\Gin@nat@height\fi}
\makeatother
% Scale images if necessary, so that they will not overflow the page
% margins by default, and it is still possible to overwrite the defaults
% using explicit options in \includegraphics[width, height, ...]{}
\setkeys{Gin}{width=\maxwidth,height=\maxheight,keepaspectratio}
\IfFileExists{parskip.sty}{%
\usepackage{parskip}
}{% else
\setlength{\parindent}{0pt}
\setlength{\parskip}{6pt plus 2pt minus 1pt}
}
\setlength{\emergencystretch}{3em}  % prevent overfull lines
\providecommand{\tightlist}{%
  \setlength{\itemsep}{0pt}\setlength{\parskip}{0pt}}
\setcounter{secnumdepth}{0}
% Redefines (sub)paragraphs to behave more like sections
\ifx\paragraph\undefined\else
\let\oldparagraph\paragraph
\renewcommand{\paragraph}[1]{\oldparagraph{#1}\mbox{}}
\fi
\ifx\subparagraph\undefined\else
\let\oldsubparagraph\subparagraph
\renewcommand{\subparagraph}[1]{\oldsubparagraph{#1}\mbox{}}
\fi

%%% Use protect on footnotes to avoid problems with footnotes in titles
\let\rmarkdownfootnote\footnote%
\def\footnote{\protect\rmarkdownfootnote}

%%% Change title format to be more compact
\usepackage{titling}

% Create subtitle command for use in maketitle
\newcommand{\subtitle}[1]{
  \posttitle{
    \begin{center}\large#1\end{center}
    }
}

\setlength{\droptitle}{-2em}
  \title{Statistical Inference Class Project}
  \pretitle{\vspace{\droptitle}\centering\huge}
  \posttitle{\par}
  \author{Cathy Snell}
  \preauthor{\centering\large\emph}
  \postauthor{\par}
  \predate{\centering\large\emph}
  \postdate{\par}
  \date{October 10, 2018}


\begin{document}
\maketitle

\section{Part 1: Simulation Exercise}\label{part-1-simulation-exercise}

\subsection{Overview}\label{overview}

we will investigate the exponential distribution in R and compare it
with the Central Limit Theorem (CLT). The CLT states that ``the
distribution of averages of iid random variables becomes that of a
standard normal as the sample size increases.'' (from 07 02 Asymptotics
and the CLT lecture transcript). We will show that \emph{need stuff
here}

\subsection{Simulations}\label{simulations}

We have been directed to investigate the distribution of averages of 40
exponentials. Some parameters have been specified for us:

\begin{itemize}
\tightlist
\item
  Simulate the exponential distribution with rexp(n, lambda)
\item
  Use lambda = 0.2 for all simulations
\item
  The mean of the exponential distribution is 1/lambda
\item
  The standard deviation of the exponential distribution is 1/lambda
\item
  Do 1000 simulations (n=1000) of 40 exponentials
\end{itemize}

(Estimate - Mean of Estimate) / Std.Err. of Estimate has a distribution
of a standard normal.

Include English explanations of the simulations you ran, with the
accompanying R code. Your explanations should make clear what the R code
accomplishes.

First, we generate a population of exponentials. We do this using the R
code provided for exponentials, and the values provided for lambda and
n. For extra fun, we generate an even larger population as well for
comparison. We expect the distributions of these to be roughtly the
same, with only the raw counts (bar heights) different.

\begin{Shaded}
\begin{Highlighting}[]
\NormalTok{pop <-}\StringTok{ }\KeywordTok{rexp}\NormalTok{(}\DecValTok{1000}\NormalTok{, .}\DecValTok{2}\NormalTok{)}
\NormalTok{lg.pop <-}\StringTok{ }\KeywordTok{rexp}\NormalTok{(}\DecValTok{2000}\NormalTok{, .}\DecValTok{2}\NormalTok{)}

\CommentTok{# combine into a data frame for plotting in ggplot2}
\NormalTok{df.pop <-}\StringTok{ }\KeywordTok{rbind}\NormalTok{(}\KeywordTok{as.data.frame}\NormalTok{(}\KeywordTok{cbind}\NormalTok{(}\DataTypeTok{Exponential=}\NormalTok{pop, }\DataTypeTok{Group=}\StringTok{"1,000 Sample"}\NormalTok{)),}
                   \KeywordTok{as.data.frame}\NormalTok{(}\KeywordTok{cbind}\NormalTok{(}\DataTypeTok{Exponential=}\NormalTok{lg.pop, }\DataTypeTok{Group=}\StringTok{"2,000 Sample"}\NormalTok{)))}

\KeywordTok{ggplot}\NormalTok{(df.pop, }\KeywordTok{aes}\NormalTok{(}\KeywordTok{as.numeric}\NormalTok{(}\KeywordTok{as.character}\NormalTok{(Exponential)))) }\OperatorTok{+}\StringTok{ }
\StringTok{  }\KeywordTok{geom_histogram}\NormalTok{(}\DataTypeTok{binwidth =} \DecValTok{5}\NormalTok{) }\OperatorTok{+}
\StringTok{  }\KeywordTok{facet_grid}\NormalTok{(.}\OperatorTok{~}\NormalTok{Group) }\OperatorTok{+}
\StringTok{  }\KeywordTok{xlab}\NormalTok{(}\StringTok{"Exponentials (binwidth=5)"}\NormalTok{)}
\end{Highlighting}
\end{Shaded}

\includegraphics{SI-project_files/figure-latex/unnamed-chunk-1-1.pdf}

Next we generate samples of 40 exponentials, 1000 times, and take the
mean of each. This is our simulation of multiple samples of the larger
population.

\begin{Shaded}
\begin{Highlighting}[]
\NormalTok{sample.means <-}\StringTok{ }\OtherTok{NULL}
\ControlFlowTok{for}\NormalTok{(i }\ControlFlowTok{in} \DecValTok{1}\OperatorTok{:}\DecValTok{1000}\NormalTok{) sample.means =}\StringTok{ }\KeywordTok{c}\NormalTok{(sample.means, }\KeywordTok{mean}\NormalTok{(}\KeywordTok{rexp}\NormalTok{(}\DecValTok{40}\NormalTok{, .}\DecValTok{2}\NormalTok{)))}
\end{Highlighting}
\end{Shaded}

\subsubsection{Sample Mean versus Theoretical
Mean}\label{sample-mean-versus-theoretical-mean}

Per the CLT, the distribution of averages/means of the samples will
approach normal, with a center/mean of the population mean.

We were given the theoretical mean of 1/lamba, which works out to 1/.2
or 5. Let's see how this compares to the mean of the populations we
simulated, and the mean of the sample means.

\begin{Shaded}
\begin{Highlighting}[]
\KeywordTok{rbind}\NormalTok{(}\KeywordTok{paste}\NormalTok{(}\StringTok{"Population (1000) mean = "}\NormalTok{, }\KeywordTok{mean}\NormalTok{(pop)),}
      \KeywordTok{paste}\NormalTok{(}\StringTok{"Population (2000) mean = "}\NormalTok{, }\KeywordTok{mean}\NormalTok{(lg.pop)),}
      \KeywordTok{paste}\NormalTok{(}\StringTok{"Means of Samples = "}\NormalTok{, }\KeywordTok{mean}\NormalTok{(sample.means)),}
      \KeywordTok{paste}\NormalTok{(}\StringTok{"Theoretical mean = "}\NormalTok{, }\DecValTok{1}\OperatorTok{/}\NormalTok{.}\DecValTok{2}\NormalTok{))}
\end{Highlighting}
\end{Shaded}

\begin{verbatim}
##      [,1]                                        
## [1,] "Population (1000) mean =  4.94951045352756"
## [2,] "Population (2000) mean =  5.12832744016011"
## [3,] "Means of Samples =  5.03207435303868"      
## [4,] "Theoretical mean =  5"
\end{verbatim}

Let's visualize the distributions with their means.

\begin{Shaded}
\begin{Highlighting}[]
\NormalTok{p1 <-}\StringTok{ }\KeywordTok{ggplot}\NormalTok{(}\KeywordTok{as.data.frame}\NormalTok{(pop), }\KeywordTok{aes}\NormalTok{(pop)) }\OperatorTok{+}\StringTok{ }
\StringTok{  }\KeywordTok{geom_histogram}\NormalTok{(}\DataTypeTok{binwidth =} \DecValTok{5}\NormalTok{, }\DataTypeTok{fill=}\StringTok{"lightblue"}\NormalTok{) }\OperatorTok{+}
\StringTok{  }\KeywordTok{xlab}\NormalTok{(}\StringTok{"Exponentials (binwidth=5)"}\NormalTok{) }\OperatorTok{+}
\StringTok{  }\KeywordTok{ggtitle}\NormalTok{(}\StringTok{"Exponential Distribution and Mean"}\NormalTok{) }\OperatorTok{+}
\StringTok{  }\KeywordTok{geom_vline}\NormalTok{(}\DataTypeTok{xintercept =} \KeywordTok{mean}\NormalTok{(pop))}

\NormalTok{p2 <-}\StringTok{ }\KeywordTok{ggplot}\NormalTok{(}\KeywordTok{as.data.frame}\NormalTok{(sample.means), }\KeywordTok{aes}\NormalTok{(sample.means)) }\OperatorTok{+}\StringTok{ }
\StringTok{  }\KeywordTok{geom_histogram}\NormalTok{(}\DataTypeTok{binwidth =}\NormalTok{ .}\DecValTok{5}\NormalTok{, }\DataTypeTok{fill=}\StringTok{"lightblue"}\NormalTok{) }\OperatorTok{+}
\StringTok{  }\KeywordTok{xlab}\NormalTok{(}\StringTok{"Means of Exponentials (binwidth=.5)"}\NormalTok{) }\OperatorTok{+}
\StringTok{  }\KeywordTok{ggtitle}\NormalTok{(}\StringTok{"Sample Means Distribution and Mean"}\NormalTok{) }\OperatorTok{+}
\StringTok{  }\KeywordTok{geom_vline}\NormalTok{(}\DataTypeTok{xintercept =} \KeywordTok{mean}\NormalTok{(sample.means))}

\KeywordTok{grid.arrange}\NormalTok{(p1, p2, }\DataTypeTok{ncol=}\DecValTok{2}\NormalTok{)}
\end{Highlighting}
\end{Shaded}

\includegraphics{SI-project_files/figure-latex/unnamed-chunk-4-1.pdf}

We can see that both the means are very close to the theoretical mean of
5.

\subsubsection{Sample Variance versus Theoretical
Variance}\label{sample-variance-versus-theoretical-variance}

We know that the standard deviation = sqrt(variance) and the theoretical
sd = 1/lambda = 5. Thus, the theoretical variance is 25.

The CLT states that the variance of the sample means will approach the
theoretical variance / n. With our sample size of 40, this gives 25/40 =
.625.

If we consider 4 different sets of sample means which vary by the number
of samples in the mean (n), we see the variance (which is the sd
squared), decline from the theoretical variance of 25.

\begin{Shaded}
\begin{Highlighting}[]
\NormalTok{sample.means2 <-}\StringTok{ }\OtherTok{NULL}
\KeywordTok{c}\NormalTok{(}\ControlFlowTok{for}\NormalTok{(i }\ControlFlowTok{in} \DecValTok{1}\OperatorTok{:}\DecValTok{1000}\NormalTok{) }\DataTypeTok{sample.means2 =} \KeywordTok{c}\NormalTok{(sample.means2, }\KeywordTok{mean}\NormalTok{(}\KeywordTok{rexp}\NormalTok{(}\DecValTok{10}\NormalTok{, .}\DecValTok{2}\NormalTok{))),}
  \ControlFlowTok{for}\NormalTok{(i }\ControlFlowTok{in} \DecValTok{1}\OperatorTok{:}\DecValTok{1000}\NormalTok{) }\DataTypeTok{sample.means2 =} \KeywordTok{c}\NormalTok{(sample.means2, }\KeywordTok{mean}\NormalTok{(}\KeywordTok{rexp}\NormalTok{(}\DecValTok{20}\NormalTok{, .}\DecValTok{2}\NormalTok{))),}
  \ControlFlowTok{for}\NormalTok{(i }\ControlFlowTok{in} \DecValTok{1}\OperatorTok{:}\DecValTok{1000}\NormalTok{) }\DataTypeTok{sample.means2 =} \KeywordTok{c}\NormalTok{(sample.means2, }\KeywordTok{mean}\NormalTok{(}\KeywordTok{rexp}\NormalTok{(}\DecValTok{30}\NormalTok{, .}\DecValTok{2}\NormalTok{))),}
  \ControlFlowTok{for}\NormalTok{(i }\ControlFlowTok{in} \DecValTok{1}\OperatorTok{:}\DecValTok{1000}\NormalTok{) }\DataTypeTok{sample.means2 =} \KeywordTok{c}\NormalTok{(sample.means2, }\KeywordTok{mean}\NormalTok{(}\KeywordTok{rexp}\NormalTok{(}\DecValTok{40}\NormalTok{, .}\DecValTok{2}\NormalTok{))))}
\end{Highlighting}
\end{Shaded}

\begin{verbatim}
## NULL
\end{verbatim}

\begin{Shaded}
\begin{Highlighting}[]
\CommentTok{# combine into a data frame for plotting in ggplot2}
\NormalTok{df.sample.means <-}\StringTok{ }\KeywordTok{rbind}\NormalTok{(}\KeywordTok{as.data.frame}\NormalTok{(}\KeywordTok{cbind}\NormalTok{(}\DataTypeTok{Sample=}\NormalTok{sample.means2[}\DecValTok{1}\OperatorTok{:}\DecValTok{1000}\NormalTok{], }\DataTypeTok{Group=}\StringTok{"10 Samples"}\NormalTok{)),}
                         \KeywordTok{as.data.frame}\NormalTok{(}\KeywordTok{cbind}\NormalTok{(}\DataTypeTok{Sample=}\NormalTok{sample.means2[}\DecValTok{1001}\OperatorTok{:}\DecValTok{2000}\NormalTok{], }\DataTypeTok{Group=}\StringTok{"20 Samples"}\NormalTok{)),}
                         \KeywordTok{as.data.frame}\NormalTok{(}\KeywordTok{cbind}\NormalTok{(}\DataTypeTok{Sample=}\NormalTok{sample.means2[}\DecValTok{2001}\OperatorTok{:}\DecValTok{3000}\NormalTok{], }\DataTypeTok{Group=}\StringTok{"30 Samples"}\NormalTok{)),}
                         \KeywordTok{as.data.frame}\NormalTok{(}\KeywordTok{cbind}\NormalTok{(}\DataTypeTok{Sample=}\NormalTok{sample.means2[}\DecValTok{3001}\OperatorTok{:}\DecValTok{4000}\NormalTok{], }\DataTypeTok{Group=}\StringTok{"40 Samples"}\NormalTok{)))}

\NormalTok{df.intercepts <-}\StringTok{ }\KeywordTok{as.data.frame}\NormalTok{(}\KeywordTok{cbind}\NormalTok{(}\DataTypeTok{Variance=} \KeywordTok{c}\NormalTok{(}\KeywordTok{var}\NormalTok{(sample.means2[}\DecValTok{1}\OperatorTok{:}\DecValTok{1000}\NormalTok{]),}
                                                \KeywordTok{var}\NormalTok{(sample.means2[}\DecValTok{1001}\OperatorTok{:}\DecValTok{2000}\NormalTok{]),}
                                                \KeywordTok{var}\NormalTok{(sample.means2[}\DecValTok{2001}\OperatorTok{:}\DecValTok{3000}\NormalTok{]),}
                                                \KeywordTok{var}\NormalTok{(sample.means2[}\DecValTok{3001}\OperatorTok{:}\DecValTok{4000}\NormalTok{])),}
                               \DataTypeTok{Group=}\KeywordTok{c}\NormalTok{(}\StringTok{"10 Samples"}\NormalTok{, }\StringTok{"20 Samples"}\NormalTok{, }\StringTok{"30 Samples"}\NormalTok{, }\StringTok{"40 Samples"}\NormalTok{)))}

\KeywordTok{ggplot}\NormalTok{(df.sample.means, }\KeywordTok{aes}\NormalTok{(}\DataTypeTok{x=}\KeywordTok{as.numeric}\NormalTok{(}\KeywordTok{as.character}\NormalTok{(Sample)))) }\OperatorTok{+}\StringTok{ }
\StringTok{  }\KeywordTok{geom_histogram}\NormalTok{(}\DataTypeTok{binwidth =}\NormalTok{ .}\DecValTok{5}\NormalTok{, }\DataTypeTok{fill=}\StringTok{"lightblue"}\NormalTok{) }\OperatorTok{+}
\StringTok{  }\KeywordTok{geom_vline}\NormalTok{(}\DataTypeTok{data =}\NormalTok{ df.intercepts, }\KeywordTok{aes}\NormalTok{(}\DataTypeTok{xintercept =} \DecValTok{5}\OperatorTok{+}\KeywordTok{as.numeric}\NormalTok{(}\KeywordTok{as.character}\NormalTok{(Variance)), }\DataTypeTok{colour=}\StringTok{"variance"}\NormalTok{)) }\OperatorTok{+}
\StringTok{  }\KeywordTok{geom_vline}\NormalTok{(}\KeywordTok{aes}\NormalTok{(}\DataTypeTok{xintercept =} \DecValTok{5}\NormalTok{, }\DataTypeTok{colour=}\StringTok{"mean"}\NormalTok{)) }\OperatorTok{+}
\StringTok{  }\KeywordTok{facet_grid}\NormalTok{(.}\OperatorTok{~}\NormalTok{Group) }\OperatorTok{+}
\StringTok{  }\KeywordTok{xlab}\NormalTok{(}\StringTok{"Sample Means"}\NormalTok{)}
\end{Highlighting}
\end{Shaded}

\includegraphics{SI-project_files/figure-latex/unnamed-chunk-5-1.pdf}

With increasing sample size, we see the spread (variance) of the normal
distribution decrease as the distribution clusters more tightly around
the mean.

\subsubsection{Distribution}\label{distribution}

We will compare the simulated populations of exponentions and the sample
means to a normal curve by plotting a normal on top of each. There are
other methods that could be used, such as the Shaprio-Wilk test, which
are out of the scope of this class.

\begin{Shaded}
\begin{Highlighting}[]
\NormalTok{p1 <-}\StringTok{ }\KeywordTok{ggplot}\NormalTok{(}\KeywordTok{as.data.frame}\NormalTok{(pop), }\KeywordTok{aes}\NormalTok{(pop)) }\OperatorTok{+}\StringTok{ }
\StringTok{  }\KeywordTok{geom_histogram}\NormalTok{(}\DataTypeTok{binwidth =} \DecValTok{5}\NormalTok{, }\DataTypeTok{fill=}\StringTok{"lightblue"}\NormalTok{, }\KeywordTok{aes}\NormalTok{(}\DataTypeTok{y=}\NormalTok{..density..)) }\OperatorTok{+}
\StringTok{  }\KeywordTok{xlab}\NormalTok{(}\StringTok{"Exponentials (binwidth=5)"}\NormalTok{) }\OperatorTok{+}
\StringTok{  }\KeywordTok{ggtitle}\NormalTok{(}\StringTok{"Exponential Distribution"}\NormalTok{) }\OperatorTok{+}
\StringTok{  }\KeywordTok{stat_function}\NormalTok{(}\DataTypeTok{fun =}\NormalTok{ dnorm, }\DataTypeTok{n =} \DecValTok{1000}\NormalTok{, }\DataTypeTok{args =} \KeywordTok{list}\NormalTok{(}\DataTypeTok{mean =} \KeywordTok{mean}\NormalTok{(pop), }\DataTypeTok{sd =} \KeywordTok{sd}\NormalTok{(pop))) }\OperatorTok{+}
\StringTok{  }\KeywordTok{ylab}\NormalTok{(}\StringTok{""}\NormalTok{) }\OperatorTok{+}
\StringTok{  }\KeywordTok{scale_y_continuous}\NormalTok{(}\DataTypeTok{breaks =} \OtherTok{NULL}\NormalTok{)}

\NormalTok{p2 <-}\StringTok{ }\KeywordTok{ggplot}\NormalTok{(}\KeywordTok{as.data.frame}\NormalTok{(sample.means), }\KeywordTok{aes}\NormalTok{(sample.means)) }\OperatorTok{+}\StringTok{ }
\StringTok{  }\KeywordTok{geom_histogram}\NormalTok{(}\DataTypeTok{binwidth =}\NormalTok{ .}\DecValTok{5}\NormalTok{, }\DataTypeTok{fill=}\StringTok{"lightblue"}\NormalTok{, }\KeywordTok{aes}\NormalTok{(}\DataTypeTok{y=}\NormalTok{..density..)) }\OperatorTok{+}
\StringTok{  }\KeywordTok{xlab}\NormalTok{(}\StringTok{"Means of Exponentials (binwidth=.5)"}\NormalTok{) }\OperatorTok{+}
\StringTok{  }\KeywordTok{ggtitle}\NormalTok{(}\StringTok{"Sample Means Distribution"}\NormalTok{) }\OperatorTok{+}
\StringTok{  }\KeywordTok{stat_function}\NormalTok{(}\DataTypeTok{fun =}\NormalTok{ dnorm, }\DataTypeTok{n =} \DecValTok{1000}\NormalTok{, }\DataTypeTok{args =} \KeywordTok{list}\NormalTok{(}\DataTypeTok{mean =} \KeywordTok{mean}\NormalTok{(sample.means), }\DataTypeTok{sd =} \KeywordTok{sd}\NormalTok{(sample.means))) }\OperatorTok{+}
\StringTok{  }\KeywordTok{ylab}\NormalTok{(}\StringTok{""}\NormalTok{) }\OperatorTok{+}
\StringTok{  }\KeywordTok{scale_y_continuous}\NormalTok{(}\DataTypeTok{breaks =} \OtherTok{NULL}\NormalTok{)}

\KeywordTok{grid.arrange}\NormalTok{(p1, p2, }\DataTypeTok{ncol=}\DecValTok{2}\NormalTok{)}
\end{Highlighting}
\end{Shaded}

\includegraphics{SI-project_files/figure-latex/unnamed-chunk-6-1.pdf}

For the exponentials, there is no left tail and we see a distinct left
skew. For the sample means, we see a clearly defined hump in the middle
with tails on either side, indicative of a normal curve.

\section{Part 2: Basic Inferential Data
Analysis}\label{part-2-basic-inferential-data-analysis}

Now in the second portion of the project, we're going to analyze the
ToothGrowth data in the R datasets package.

\subsection{Exploritory Data Analysis}\label{exploritory-data-analysis}

Load the ToothGrowth data and perform some basic exploratory data
analyses

Provide a basic summary of the data.

\subsection{Use confidence intervals and/or hypothesis
tests}\label{use-confidence-intervals-andor-hypothesis-tests}

to compare tooth growth by supp and dose. (Only use the techniques from
class, even if there's other approaches worth considering)

\subsection{Conclusion}\label{conclusion}

State your conclusions and the assumptions needed for your conclusions.


\end{document}
